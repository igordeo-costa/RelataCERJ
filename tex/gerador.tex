\documentclass[a4paper,11pt]{article}
\newif\ifrelatoriocomplexo % Tipo de relatório, normal ou confidencial
\relatoriocomplexofalse    % Para gerar relatório simples (false, opção default) ou complexo (true)

\usepackage{fontspec}
\setmainfont{IBM Plex Sans}
\usepackage{microtype}
\usepackage{polyglossia}
\setdefaultlanguage[variant=brazilian]{portuguese}
\usepackage{fullpage}
\usepackage{amsmath}
\usepackage{amssymb}
\usepackage{graphicx}
\usepackage{eso-pic}
\usepackage{ragged2e}
\usepackage{datatool}
\usepackage{expl3}
\usepackage{luacode}
\usepackage{xstring}
\usepackage{xcolor}
\usepackage{xparse}
\usepackage{seqsplit}
\usepackage{transparent}
\usepackage{multicol}

\DTLloaddb{tabela}{../data/DadosBrutos.csv}

%%% Definindo path para deixar script mais limpo
\makeatletter
  \def\input@path{{tex/includes/}}
\makeatother

%%% Includes
%%% ----------------------------------------------------
%%% Contar participantes a partir da lista de nomes
%%% ChatGPT que fez pra mim ;)
\ExplSyntaxOn

\NewDocumentCommand{\ContaParticipantes}{m}
 {
   \__conta_participantes:n { #1 }
 }

\cs_new_protected:Npn \__conta_participantes:n #1
 {
   % remove aspas duplas
   \tl_set:Nn \l_tmpa_tl { #1 }
   \tl_replace_all:Nnn \l_tmpa_tl { " } { }

   % cria uma comma list a partir do texto (com expansão!)
   \clist_set:Nx \l_tmpa_clist { \l_tmpa_tl }

   % conta os elementos
   \int_eval:n { \clist_count:N \l_tmpa_clist }
 }

\ExplSyntaxOff
%%% ----------------------------------------------------


%%% Código para cálculo do tempo de duração da atividade
%%% usa linguagem "lua", que não domino.
%%% Feito com ajuda do DeepSeek

\begin{luacode}
function duracao(inicio, hora_inicio, fim, hora_fim)
    -- Função para analisar data e hora
    local function parse_datetime(date_str, time_str)
        local day, month, year = date_str:match("(%d+)/(%d+)/(%d+)")
        local hour, minute, second = time_str:match("(%d+):(%d+):(%d+)")
        
        return os.time({
            year = tonumber(year),
            month = tonumber(month),
            day = tonumber(day),
            hour = tonumber(hour),
            min = tonumber(minute),
            sec = tonumber(second)
        })
    end
    
    -- Se fim for vazio, usar início
    if fim == "" then
        fim = inicio
    end
    
    local t0 = parse_datetime(inicio, hora_inicio)
    local t1 = parse_datetime(fim, hora_fim)
    
    local diff_seconds = os.difftime(t1, t0)
    
    -- Calcular dias, horas e minutos
    local dias = math.floor(diff_seconds / 86400)  -- 86400 segundos em um dia
    local segundos_resto = diff_seconds % 86400
    local horas = math.floor(segundos_resto / 3600)
    local minutos = math.floor((segundos_resto % 3600) / 60)
    
    -- Formatar a saída de acordo com a duração
    if dias == 0 then
        -- Menos de 1 dia: mostrar horas e minutos
        if minutos > 0 then
            -- Verificar plurais para horas
            local hora_texto = (horas == 1) and "hora" or "horas"
            -- Verificar plurais para minutos
            local minuto_texto = (minutos == 1) and "minuto" or "minutos"
            return string.format("%d %s e %d %s", horas, hora_texto, minutos, minuto_texto)
        else
            -- Apenas horas
            local hora_texto = (horas == 1) and "hora" or "horas"
            return string.format("%d %s", horas, hora_texto)
        end
    else
        -- 1 dia ou mais: mostrar dias e horas
        if horas > 0 then
            -- Verificar plurais para dias
            local dia_texto = (dias == 1) and "dia" or "dias"
            -- Verificar plurais para horas
            local hora_texto = (horas == 1) and "hora" or "horas"
            return string.format("%d %s e %d %s", dias, dia_texto, horas, hora_texto)
        else
            -- Apenas dias
            local dia_texto = (dias == 1) and "dia" or "dias"
            return string.format("%d %s", dias, dia_texto)
        end
    end
end
\end{luacode}

%%% Comando para cálculo do trimestre
\begin{luacode}
function trimestre_atual()
    local mes = tonumber(os.date("%m"))
    local ano = os.date("%Y") -- ano está o do sistema operacional (mudar para data da última excursão)

    -- Divide o mês por três e arredonda para cima (ceil)
    -- Ex. Abril é mês 4/3 = 1,33 = 2º Trimestre
    local trimestre = math.ceil(mes / 3)

    local ordinais = utf8.char(0x00BA) -- Símbolo "º"

    return string.format(
      "%d%s Trimestre de %s",
      trimestre,
      ordinais,
      ano
    )
  end
\end{luacode}

\setlength\parindent{24pt}

\usepackage{fancyhdr}

\fancypagestyle{fancy}{
	%Headers
	\fancyhead[L]{}
	\fancyhead[C]{}
	\fancyhead[R]{}
\renewcommand{\headrulewidth}{0pt}
	%Footers
	\fancyfoot[L]{\footnotesize Diretoria Técnica do CERJ -- Relatório de Excursões}
	\fancyfoot[C]{}
	\fancyfoot[R]{\footnotesize Relatório Nº: 2026.\therelatorio}
\renewcommand{\footrulewidth}{0.0pt}
}
\headheight=0pt
\footskip=35pt

\makeatletter
\let\ps@plain\ps@fancy
\makeatother

\pagestyle{fancy}

% Colors for links (customizable)
\usepackage{xcolor}
\definecolor{linkblue}{HTML}{0050A4}   % deep but non-aggressive blue
\definecolor{citegreen}{HTML}{207A39}  % elegant green
\definecolor{urlmagenta}{HTML}{8B1C62} % strong, readable purple

% Hyperref (load LAST)
\usepackage[
    unicode=true,
    pdfencoding=auto,
    pdfpagelabels=true,
    psdextra,                  % safer unicode bookmarks
    bookmarks=true,
    bookmarksnumbered=true,
    hyperfootnotes=true,
    breaklinks=true,
    pdfborder={0 0 0},         % no ugly boxes
    linktoc=all,               % TOC entries clickable
    colorlinks=true,
    linkcolor=linkblue,
    citecolor=citegreen,
    urlcolor=urlmagenta,
    filecolor=black,
    pdfstartview=FitV,         % nice initial zoom
    pdfpagemode=UseNone,       % no sidebar by default
    pdfhighlight=/I,           % “invert” highlight on click
]{hyperref}

% METADATA (automatic from commands)
\hypersetup{
    pdftitle    = {Relatório de Excursões Oficiais -- CERJ 2026},
    pdfauthor   = {Diretoria Técnica do CERJ},
    pdfsubject  = {Formulário Institucional},
    pdfkeywords = {Relatório, Cadastro, Formulário},
    pdfcreator  = {LuaLaTeX with hyperref},
}


%%% -----------------------------------------------------------------------
%%% Comando para cálculo de duração da atividade
\newcommand{\calcularDuracao}[4]{%
    \directlua{tex.sprint(duracao("#1", "#2", "#3", "#4"))}%
}

%%% Comando para cálculo do trimestre na capa do relatório
\newcommand{\trimestreAtual}{\directlua{tex.print(trimestre_atual())}}

%%% Comando para apresentar a hora no formato hh:mm e não hh:mm:ss
\newcommand{\formatarHora}[1]{%
  \StrBefore{#1}{:}:%
  \StrBetween[1,2]{#1}{:}{:}%
}

%%% Comando para exibição dos participantes empilhados no tabular
\newcommand{\ParticipantesEmColunas}[1]{%
  \begin{minipage}[t]{\linewidth}
    \setlength{\columnsep}{15pt}%
    \begin{multicols}{3}
      \raggedright
      \noindent
      \StrSubstitute{#1}{, }{\\}%
    \end{multicols}
  \end{minipage}%
}

%%% -----------------------------------------------------------------------

%%% Capa do documento
\newcommand{\TituloRelatorio}{RELATÓRIOS \par de Excursões Oficiais}
\newcommand{\SubtituloRelatorio}{Centro Excursionista Rio de Janeiro}
\newcommand{\TrimestreRelatorio}{\trimestreAtual}

%%% Cor do título na capa
\definecolor{azulescuro}{RGB}{0,51,102}

%%% comando para organização do sumário
\newcommand{\EntradaSumarioExcursao}[1]{%
  \addcontentsline{toc}{section}{\excursao\ (\guia)}%
}
%%% -----------------------------------------------------------------------

%%% Contador do número do relatório
\newcounter{relatorio}

%%% O documento
\begin{document}
\frenchspacing
\pagenumbering{gobble} % não mostra número de página

%%% Capa
\thispagestyle{empty}

\AddToShipoutPictureBG*{%
  \AtPageLowerLeft{%
  \put(0,0){%
  %\AtPageCenter{%
  %\raisebox{-8cm}{%
  \transparent{0.06}%
    \includegraphics[
      width=0.75\paperwidth]{../img/LogoRetro1939.png}%
  }%
}%
}

\begin{center}
  \vspace*{3cm}
  {\Huge \bfseries\color{azulescuro} \TituloRelatorio \par}
  \vspace{1cm}
  {\Large \bfseries\color{azulescuro} \SubtituloRelatorio \par}
  \vfill
  {\large \color{azulescuro} \TrimestreRelatorio}
\end{center}

\clearpage

%%% Sumário
\tableofcontents
\thispagestyle{empty} % Garante página do sumário sem rodapé
\clearpage
\pagenumbering{arabic}
\setcounter{page}{1}

%%% Iniciar o contador aqui para não afetar o sumário!
\setcounter{relatorio}{1}

\DTLforeach{tabela}{
  \excursao=4,
  \local=12,
  \guia=2, \segundoGuia=15,
  \auxiliar=3,
  \tipo=5,
  \inicio=6, \fim=7,
  \horaInicio=8, \horaFim=14,
  \clima=9,
  \listaParticipantes=10,
  \observacoes=11,
  \confidencial=16,
  \relator=13, \dataRelato=1}{

  \AddToShipoutPictureBG*{%
    \AtPageUpperLeft{%
      \put(\LenToUnit{\paperwidth - 5.5cm}, -4.5cm){%
        \includegraphics[width=3cm]{../img/logo.png}%
      }%
    }%
  }

  \EntradaSumarioExcursao{\excursao} % Para gerar o sumário
  \section*{\excursao}
  \vspace{-0.8\baselineskip}
    \DTLifstringeq{\fim}{} % Data e hora de início e fim da atividade
  		{\inicio, de \formatarHora{\horaInicio}\ até \formatarHora{\horaFim}}
  	{
    		\DTLifstringeq{\inicio}{\fim}
      			{\inicio, de \formatarHora{\horaInicio}\ até \formatarHora{\horaFim}}
      			{de \inicio, às \formatarHora{\horaInicio} \\ até \fim, às \formatarHora{\horaFim}}
  	} \\

%%% Marcação de relatório confidencial em vermelho!
\ifrelatoriocomplexo
\noindent\textbf{\textcolor{red}{CONFIDENCIAL!}} \\
\textcolor{red}{Este relatório contém informações críticas e/ou sensíveis! \\ Leitura exclusiva da Diretoria Técnica.}
\fi

\section*{\null}
  \vspace{-2\baselineskip}
  \begin{tabular}{@{} l p{12cm} @{}}
    \textbf{Local:}           & \local \\
    \textbf{Classificação:} 	& \tipo \\
    \textbf{Clima:} 		      & \clima \\
    \textbf{Duração:} 		    & \calcularDuracao{\inicio}{\horaInicio}{\fim}{\horaFim} \\
    \textbf{Guia(s):} 		    & \DTLifstringeq{\segundoGuia}{}% trata da exibição de um segundo guia na atividade
  						                                {\guia}%
  						                                {\guia\ e\ \segundoGuia} \\
    \textbf{Auxiliar(es):} 	  & \auxiliar \\
     \textbf{Participante(s):}& \ContaParticipantes{\listaParticipantes}
  \end{tabular}

 % Lista de participantes em colunas
  \begin{tabular}{@{} l p{12cm} @{}}
  	\ParticipantesEmColunas{\listaParticipantes}
  \end{tabular}

% Observações e relato do guia sobre a excursão
\subsubsection*{Relato, Observações e outros Comentários:}

  \begin{tabular}{@{} l p{15cm} @{}}
    & \observacoes \\
  \end{tabular}

% Observações críticas e ou sensíveis sobre a excursão
% Gerar essa seção apenas para apreciação do DT quando necessário
\ifrelatoriocomplexo
\subsubsection*{Observações críticas e/ou sensíveis sobre a excursão:}

	\begin{tabular}{@{} l p{15cm} @{}}
	 		& \edef\confidencialtmp{\confidencial} % Normaliza a coluna do .csv se estiver com espaço e não vazia de fato
	 			\DTLifstringeq{\confidencialtmp}{}
  					{\textcolor{red}{Não há informações críticas ou sensíveis neste relatório!}}
  					{\textcolor{red}{\confidencial}} \\
  	\end{tabular}
\fi

\subsubsection*{\null}
 \begin{tabular}{@{} l p{15cm} @{}}
    \textbf{Relatório enviado por:} & \relator\ em \dataRelato \\
  \end{tabular}

  \vfill

  \pagebreak
  \stepcounter{relatorio}
}

\end{document}

% Local Variables:
% jinx-languages: "pt_BR"
% End:
